\documentclass[12pt,a4paper]{article}
\setlength\textwidth{145mm}
\setlength\textheight{247mm}

\usepackage[a-2u]{pdfx}

%% Character encoding: usually latin2, cp1250 or utf8:
\usepackage[utf8]{inputenc}

%% Prefer Latin Modern fonts
\usepackage{lmodern}

%% Further useful packages (included in most LaTeX distributions)
\usepackage{amsmath}        % extensions for typesetting of math
\usepackage{amsfonts}       % math fonts
\usepackage{amsthm}         % theorems, definitions, etc.
\usepackage{bbding}         % various symbols (squares, asterisks, scissors, ...)
\usepackage{bm}             % boldface symbols (\bm)
\usepackage{graphicx}       % embedding of pictures
\usepackage{fancyvrb}       % improved verbatim environment
%\usepackage{natbib}         % citation style AUTHOR (YEAR), or AUTHOR [NUMBER]
\usepackage[style=ieee]{biblatex}
\usepackage[nottoc]{tocbibind} % makes sure that bibliography and the lists
\usepackage{dcolumn}        % improved alignment of table columns
\usepackage{booktabs}       % improved horizontal lines in tables
\usepackage{paralist}       % improved enumerate and itemize
\usepackage[usenames]{xcolor}  % typesetting in color
\usepackage{float}          % putting images at desired place

\hypersetup{unicode}
\hypersetup{breaklinks=true}
\DefineVerbatimEnvironment{code}{Verbatim}{fontsize=\small, frame=single}
\newcommand{\R}{\mathbb{R}}
\newcommand{\N}{\mathbb{N}}
\begin{document}
\title{SnowPlow – specification}
\author{Ondřej Měkota}

\maketitle
\pagebreak
%\tableofcontents
%\pagebreak
\section{Introduction}

\par This program is a plugin to program \href{https://qgis.org/en/site/index.html}{QGIS 3.4}. 
Its main function is ease the usage of QGIS with certain types of maps.
Firstly it colours edges according to specified rules.
Also it provides the user with statistics of the data.


\section{Basic specification}
\begin{itemize}
    \item Selection of layer to which the plugin will be applied.
    \item Make it horizontally layed out.
    \item On the right there will be settings – layer select, columns select for stats, function (sum, avg, ...) to apply on the remaining columns.
    \item The function could be different for every column.
    \item Button to invoke colourization and labeling. Resets current colourization and makes new one for selected layer.
    \item Highlighting all transits of selection of cars – roads, which is not maintained by the given car. 
        \emph{note} This will be done by event onClick on given car in listwidget or writing the car ID)
        Deleting can be done by clicking on a car in a listwidget of selected cars.
        Each transit can have symbol denoting the transiting car.
    \item Plugin should distinguish data types of layers, ie. not summing over textual data. Also is should advise the user not to select columns with high number of disctinct values, eg. float type.
        

\end{itemize}

\end{document}
