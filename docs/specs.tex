\documentclass[12pt,a4paper]{article}
\setlength\textwidth{145mm}
\setlength\textheight{247mm}

\usepackage[a-2u]{pdfx}

%% Character encoding: usually latin2, cp1250 or utf8:
\usepackage[utf8]{inputenc}

%% Prefer Latin Modern fonts
\usepackage{lmodern}

%% Further useful packages (included in most LaTeX distributions)
\usepackage{amsmath}        % extensions for typesetting of math
\usepackage{amsfonts}       % math fonts
\usepackage{amsthm}         % theorems, definitions, etc.
\usepackage{bbding}         % various symbols (squares, asterisks, scissors, ...)
\usepackage{bm}             % boldface symbols (\bm)
\usepackage{graphicx}       % embedding of pictures
\usepackage{fancyvrb}       % improved verbatim environment
%\usepackage{natbib}         % citation style AUTHOR (YEAR), or AUTHOR [NUMBER]
\usepackage[style=ieee]{biblatex}
\usepackage[nottoc]{tocbibind} % makes sure that bibliography and the lists
\usepackage{dcolumn}        % improved alignment of table columns
\usepackage{booktabs}       % improved horizontal lines in tables
\usepackage{paralist}       % improved enumerate and itemize
\usepackage[usenames]{xcolor}  % typesetting in color
\usepackage{float}          % putting images at desired place

\hypersetup{unicode}
\hypersetup{breaklinks=true}
\DefineVerbatimEnvironment{code}{Verbatim}{fontsize=\small, frame=single}
\newcommand{\R}{\mathbb{R}}
\newcommand{\N}{\mathbb{N}}
\begin{document}
\title{SnowPlow – specification}
\author{Ondřej Měkota}

\maketitle
\pagebreak
\tableofcontents
\pagebreak
\section{Introduction}

\par This program is a plugin to program \href{https://qgis.org/en/site/index.html}{QGIS 3.4}. 
Its main function is ease the usage of QGIS with certain types of maps.
Firstly it colours edges according to specified rules.
Also it provides the user with statistics of the data.


\section{Specification}
\begin{itemize}
    \item Selection of layer to which the plugin is applied.
    \item On the left of the window there settings – layer selection, row selection for the statistics, function (sum, avg, max, min) to apply on the remaining columns.
    \item The function can be different for every column.
    \item There is button to invoke colourization and labeling. It resets current colourization and makes new one for selected layer.
    \item Highlighting all (at once, or only selected) transits of selection of cars – roads, which are not maintained by the given cars but the cars use them for transport. 
    \item Plugin distinguishes data types of features in layers, ie. not summing over textual data. 
\end{itemize}

\pagebreak
\section{User Manual}
\subsection{Installation}
First install QGIS version 3.4 or higher. 
Then in "Plugin" $\rightarrow$ "Manage and install plugins" $\rightarrow$ "Install from ZIP" $\rightarrow$ \emph{select the path of the zip file} $\rightarrow$ "Install Plugin".
The plugin shall appear in the menu "Plugins".

\subsection{Using the application}
The plugin has one window, depicted in Figure \ref{window}, which contains all controls.

\begin{figure}[H]\centering
\includegraphics[width=140mm]{./img/window.png}
\caption{Description of user interface}
\label{window}
\end{figure}

\subsubsection{Diplaying statistics}
After selecting rows for grouping, and optionally selecting grouping function, user will be presented with table containing statistics.
Data are grouped by cartesian product of those selected features for grouping. 

\end{document}
